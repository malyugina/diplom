\chapter{Обзор литературы}

\section{Вариабельность сердечного ритма }

Вариабельность сердечного ритма (Heart rate variability, HRV) это явление изменения частоты сердечного ритма.

Вариабельность сердечного ритма сильно зависит от эмоционального \cite{hrv_and_sensitivity, hrv_and_respiratory} и физического \cite{hrv_and_phisical_health} здоровья человека и может использоваться как его показатель.

Существуют два основных причины колебания частоты сердечного ритма \cite{two_rates_hrv}.

\begin{itemize}
	\item Колебания, вызванные дыхательной аритмией. При этом частота сердечных сокращений изменяется в связи с дыханием и можно отследить частоту дыхания.
	\item Низкочастотные колебания артериального давления. Это явление называется волнами Майера \cite{mayer_wave} артериального давления и, как правило, имеют частоты порядка 0,1 Гц.
\end{itemize}

Для наблюдения за сердечным ритмом применяются следующие методы:

\begin{itemize}
	\item ЭКГ (электрокардиография) \cite{EKG};
	\item баллистокардиография  \cite{Ballistocardiograms};
	\item фотоплетизмография \cite{Photoplethysmography}.
\end{itemize}

ЭКГ считается методом, дающим наиболее четкий сигнал.

\subsection{ЭКГ}

Электрокардиограмма – это запись колебаний разности потенциалов, возникающих на поверхности возбудимой ткани или в окружающей сердце проводящей среде при распространении волны возбуждения по сердцу.\cite{ekg1} Запись ЭКГ производится с помощью электрокардиографов и различных систем отведений ЭКГ. Каждое отведение регистрирует разность потенциалов, существующую между двумя определенными точками электрического поля сердца, в которых установлены электроды. Пример ЭКГ можно увидеть на рис. \ref{ris:RR}.

\begin{figure}[h]
	\begin{center}
		\includegraphics[scale=0.5]{real_ekg}
		\caption{ЭКГ}
		\label{ris:RR}
	\end{center}
\end{figure}
\subsubsection{Немного об устройстве сердца}

Сердце обладает рядом свойств.

\begin{itemize}
	\item Свойство автоматизма - это способность сердца вырабатывать электрические импульсы при отсутствии внешних раздражений.
	\item Свойство проводимости - это способность к проведению возбуждения волокон проводящей системы сердца и сократительного миокарда.
	\item Свойство возбудимости – это способность клеток проводящей системы сердца и сократительного миокарда возбуждаться под влиянием внешних электрических импульсов.
\end{itemize}

Возбуждение сердечной мышцы сопровождается возникновением изменяющейся разности потенциалов между наружной и внутренней поверхностью клеточной мембраны сердца.

При распространении по сердцу волны деполяризации наружная поверхность клетки приобретает отрицательный заряд, а во время реполяризации – положительный. Согласно концепции В. Эйнтховена, сердце в каждый момент сердечного цикла можно рассматривать как точечный единый диполь, который создает в окружающей его среде электрическое поле. Положительный полюс диполя (+) всегда обращен в сторону невозбужденного, а отрицательный полюс (–) – в сторону возбужденного участка сердца(рис. \ref{ris:dipol}).

\begin{figure}[h]
	\begin{center}
		\includegraphics{heart-dipol}
		\caption{Поле сердечного диполя в отведениях}
		\label{ris:dipol}
	\end{center}
\end{figure}


Помещая положительный и отрицательный электроды в любые точки электрического поля, можно зарегистрировать разность потенциалов, существующую между этими точками в каждый момент деполяризации и реполяризации сердца. Конфигурация такой ЭКГ прежде всего будет зависеть от направления вектора диполя по отношению к электродам.

В настоящее время в клинической практике наиболее широко используют 12 отведений ЭКГ, запись которых является обязательной при каждом электрокардиографическом обследовании больного: 3 стандартных отведения, 3 усиленных однополюсных отведения от конечностей и 6 грудных отведений.

Стандартные отведения от конечностей регистрируют при следующем попарном подключении электродов:

\begin{itemize}
	\item I отведение – левая рука (+) и правая рука (–);
	\item II отведение – левая нога (+) и правая рука (–);
	\item III отведение – левая нога (+) и левая рука (–).
\end{itemize}

\subsection{RR}
Если рассматривать момент сокращения сердца более подробно \cite{polarization_heart} виден QRS-комплекс - изменения электрического потенциала, предшествующее сокращению сердца. За электрической активацией клеток (деполяризацией) следует механическое сокращение. Процесс сокращения сердечных мыщц состоит из следующих стадий (рис.\ref{ris:RR}).

\begin{itemize}
	\item деполяризация предсердий; зубец P
	\item передача импульса желудочкам; интервал PR
	\item деполяризация желудочков; зубец R
	\item реполярицазия желудочков; зубец T
	\item мнения исследователей относительно причин происхождения зубца V различны.
\end{itemize}

\begin{figure}[h]
	\begin{center}
		\includegraphics[scale = 0.5]{RR.png}
		\caption{ЭКГ}
		\label{ris:RR}
	\end{center}
\end{figure}

Для различных отведений вид QRS комплекс может отличаться. Вместо пика R может быть глубокий минимум. Минимумы Q, S могут отсутствовать. Приведенный на рисунке выше и на рис. примеры соответствуют первому отведению.

По длительности интервалов, глубине минимумов и высоте максимумов врачи предсказывают различные характеристики и болезни.

RR-интервалом называется промежуток времени между R пиками (последовательными ударами сердца). Нормально частотой серцебиения считается 60-100 ударов в минуту в состоянии покоя. Ниже приведены примеры RR-сигнала (последовательность RR-интервалов) (рис. \ref{real_RR}).

\begin{figure}[h]
	\begin{minipage}[h]{0.47\linewidth}
		\center{\includegraphics[width=1\linewidth]{rr_100}} a) \\
	\end{minipage}
	\hfill
	\begin{minipage}[h]{0.47\linewidth}
		\center{\includegraphics[width=1\linewidth]{rr_200}} \\b)
	\end{minipage}
	\vfill
	\begin{minipage}[h]{0.47\linewidth}
		\center{\includegraphics[width=1\linewidth]{rr_500}} c) \\
	\end{minipage}
	\hfill
	\begin{minipage}[h]{0.47\linewidth}
		\center{\includegraphics[width=1\linewidth]{rr_1000}} d) \\
	\end{minipage}
	\begin{minipage}[h]{0.47\linewidth}
		\center{\includegraphics[width=1\linewidth]{rr_10000}} e) \\
	\end{minipage}
	\hfill
	\begin{minipage}[h]{0.47\linewidth}
		\center{\includegraphics[width=1\linewidth]{rr_70000}} f) \\
	\end{minipage}
	\caption{RR-сигнал длительностью a)100 отсчетов $\sim$ 70с) b) 200 отсчетов
		c) 500 отсчетов d) 1000 отсчетов e)10000 отсчетов ($\sim $2 часа) f)70000 отсчетов.}
	\label{ris:real_RR}
\end{figure}


\subsection{Анализ RR сигнала}
После фильтрации ЭКГ и выделения R-пиков мы получаем их последовательность. В данной главе речь пойдет об работе с RR-сигналом - последовательностью промежутков времени между R-пиками (RR-интервалов). Ка было сказано выше, мы выкинули из нашего сигнала интервалы более 2 секунд и менее 2 милисекунд. 

Существует перечень стандартных признаков, которые считаются информативными и обычно используются при анализе RR-сигнала.
Их можно разделить на несколько групп.

\begin{enumerate}
	\item Признаки, основанные на подсчете стандартных статистик анализе RR интервалов. При данном анализе часто подсчитываются следующие характеристики \cite{pNN, metric_of_hrv}.
	прочитать проверить !!!! хрень таки
	\begin{enumerate}
		\item SDNN - стандартное отклонение интервалов от среднего, вычесленого почти всегда в течение 24 часов. SDNN обычно вычисляется для коротких интервалов, порядка 5 минут. SDNN отражает все циклические компоненты, ответственные за изменчивость в период записи, поэтому она представляет собой общую вариабельность.
		\item 	RMSSD ("среднеквадратичное последовательных различий») - квадратный корень из среднего значения квадратов последовательных RR-интервалов.
		\item SDSD ("стандартное отклонение последовательных разностей") - стандартное отклонение последовательных разностей между соседними .
		\item NN50, количество пар последовательных NN ударов, которые отличаются более чем 50 мс.
		\item pNN50, доля NN50 делится на общее количество NN ударов.
		\item 	NN20, количество пар последовательных NN ударов, которые отличаются более чем 20 мс.
		\item pNN20, доля NN20.
		\item EBS ("оценивается цикл дыхания"), диапазон (макс-мин) в скользящем окне заданной длительности. Окна можно перемещать без перекрытия или с перекрытием. ЕВС часто используют при сборе данных, где обратная связь ВСР в режиме реального времени является главной целью.
	\end{enumerate}
	\item Геометрические методы \cite{geometric_metric} - дописать)
	\item Частотный анализ
	
	Строятся гистограммы частотных характеристик NN ударов
	
	\item Подсчет кореляций следующих NN ударов от предидущих \cite{autocorr_metric}
	\item Нелинейные методы \cite{non_linear_metric}
	
	\begin{enumerate}
		\item график Пуанкаре \cite{poinkare_plot}.
		\item фрактальные размерности \cite{fractal_dim}.
		\item флуктуационный анализ \cite{fluct_analis}
		\item энтропия \cite{entropy1, entropy2, entropy3}
		\item и др. \cite{other_analis1, other_analis2, other_analis3}
	\end{enumerate}	
\end{enumerate}

\section{Предсказание сна}
Актуальность исследований сна и ее психологических факторов обусловлена, в первую очередь, практической необходимостью. 
Расстройства сна, такие как бессоница, которая по данным эпидемиологии встречается у 28-45\% населения, в половине случаев является важной проблемой, требующей лечения \cite{sleep3}.
При этом 1.5-3\% населения постоянно и 25-29\% эпизодически принимают снотворные препараты. 
У больных бессоницей выше риск психических и соматических заболеваний и смертность. Существующие исследования показали, чтоу людей, страдающих бессонницей, риск развития гипертонии на 350\% выше, чем у обычных людей. 
Бессонница также является фактором риска для сахарного диабета, тревоги и депрессии.
Среди пожилых людей до 35\% населения имеют проблемы со сном, в именно до 20\% мужчин и до 50\% женщин. \cite {sleep1, sleep2}
Для точной оценки бессоницы необходимы замеры длительности сна, основанные на данных полисомнографической системы. 
Однако эти системы неудобны так для проведения замеров, нобходимо провести ночь в медицинском центре в специальных установках.

Сейчас исследуются новые методы наблюдения за сном \cite{monitor_sleep, monitor_sleep2, monitor_sleep3}. Интерес в портативных устройствах в данной сфере постоянно растет.
Недостатком предложенных методов является необходимость в дополнительном оборудовании. Однако при этом при определение состояния сон/бодроствование не было  сследований, в которых получены достаточно точные результаты.

\section{деревья,логистическая регресия, градиентный бустинг на деревьях - методы машинного обучения}

\section{Рекуррентные нейронные сети}
Рекуррентная нейронная сеть – один из типов нейронных сетей \cite{neural_network}, в котором присутствует обратная связь. Другими словами в выход более позднего слоя сети поступает на вход более слоя, считающегося ранее.

Рекуррентные нейронные сети имеют ряд преимуществ перед обычными.
\begin{itemize}
	\item Разнообразные виды входных данных
	\item Способность строить неявные модели данных
	\item Устойчивость к выбросам
	\item 
\end{itemize}

Рекуррентные сети применяются в таких задачах как:
\begin{itemize}
	\item распознавание устной и письменной речи \cite{rnn_for_speech_recognition, rnn_for_text_recognition};
	\item машинный перевод \cite{rnn_for_translation};
	\item предсказания временных рядов \cite{rnn_for_prediction};
	\item И т.п.
\end{itemize}



Основной проблемой при использовании рекуррентных нейронных сетей является проблема исчезающего градиента (vanishing gradient problem \cite{vanishing_gradient_problem}). Если честно подсчитывать градиент на каждой итерации обучения он будет выражаться через произведение градиентов, подсчитанных на всех предыдущих итерация. Если градиенты были малы – наш градиент будет просто не заметен и обучение встанет надолго. 

Проблема исчезающего градиента была подробно рассмотрена Хочрайтером???(произношение) and Шмидхубера, которые разработали LTSM архитектуру \cite{create_lstm}, которая устойчива к данной проблеме. Данная архитектура оказалась проста в использовании и стала стандартным средством борьбы с исчезающим градиентом. Также были осуществлены попытки других способов решения данной проблемы:
\begin{itemize}
	\item обратное распостранение во времени (Backpropagation through time) \cite{Backpropagation}. Используется для обучения сетей Элмана.
	\item использование мощного secondorder алгоритмы оптимизации (Martens, 2010; Мартенс Sutskever, 2011)
	\item регуляризация весов в RNN, что гарантирует, что градиент не обращается в нуль (Pascanu др.,2012)
	\item отказ от всего обучения текущие веса (Jaeger, Haas, 2004; Jaeger, 2001), и очень
	осторожны инициализации параметров РНН (в Sutskever др.,2013).
\end{itemize}
	
Также была обнаружена проблема сильного возрастание градиента, но она легко решилась ограничение на модуль градиента.

\subsection{LSTM}

Данный слой успешно борется с проблеммай исчезающего градиента. Каждый нейрон в данном слое представняет собой "ячейку памяти" (рис. \ref{ris:ltsm}). Данные ячейки организованны для хранения информации и позволяют более качествеено обрабатывать длинные входные последовательности.

\begin{figure}[h]
\begin{center}
	\includegraphics[scale=0.36]{ltsm.png}
	\caption{LSTM ячейка}
	\label{ris:ltsm}
\end{center}
\end{figure}


Пускай на вход LSTM ячейке памяти подается часть входной последовательности $x_t = (x_1, ..., x_N)$ и предыдущие выходной вектор признаков $h_{t-1} = (h_1, ..., h_M)$. Ячейка должна подсчитать вектор признаков $h_t$. Ячейка организованны следующим обрахом:


\begin{equation}
	i_i = \sigma(W_{xi}*x_t+W_{hi}*h_{t-1}+W_{ci}*c_{t-1}+b_i)
\end{equation}
\begin{equation}
	f_t = \sigma(W_{xf}*x_t+W_{hf}*h_{t-1}+W_{cf}*c_{t-1}+b_f)
\end{equation}
\begin{equation}
	c_t = f_t*c_{t-1}+i_t*\tanh(W_{xc}*x_{t-1}+W_{hc}*h_{t-1}+b_c)
\end{equation}
\begin{equation}
	o_t = \sigma(W_{xo}*x_t+W_{ho}*h_{t-1}+W_{co}*c_t+b_o)
\end{equation}
\begin{equation}
	h_t = o_t*\tanh(c_t)
\end{equation}

Здесь $\sigma = \frac{1}{1+e^{-x}} $ - сигмоида, логистическая функция, принимающая значения от 0 до 1. А i, f, o и c - вход ячейки, клетка памяти, выход ячейки и клетка активации, по содержимому которой решается, будет ли использоваться ектор из клетки памяти. Все вышеперечисленные вектора имеют размерность $h_t$.

