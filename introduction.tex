\chapter{Введение}

Предсказанием и распознаванием болезней человека, его различных характеристик по сигналу ЭКГ занимаются многие группы ученых \cite{about_ekg_sciense1, about_ekg_sciense2}. Медицинские институты собирают данные, ученые исследуют уже известные методы получения и обработки информации об сердечной активности и изобретают новые \cite{get_ekg_signal}. Благодаря развитию медицинских технологий, позволяющих получать ЭКГ сигнал круглосуточно, задачи в этой области становятся все более актуальными. Множество людей сегодня носят фитнес-браслеты, наблюдающие за пульсом человека, создаются модели холтеров, позволяющие получать все более точный сигнал, продаются чехлы для телефонов, способные записывать ЭКГ сигнал в любой, удобный для вас момент времени.

Анализируя эту информацию, можно вовремя распознать различные болезни и принять предупредительные меры. Также одной из важных задач анализа ЭКГ сигнала является определения периодов сна/бодрствования человека. Сонливое состояние в течение дня - одна из главных причин несчастных случаев на дорогах и производстве \cite{accidents}. Как следствие большой интерес представляет осуществление постоянного контроля за уровнем сонливости лиц, чья работа связана с риском, таких как летчики, водители-дальнобойщики и др.. Предубеждение В настоящее время в промышленности для предотвращение чрезвычайных ситуаций используется наблюдение за поведением операторов (открыты ли глаза). 

В данной работе рассказано о способах фильтрации ЭКГ-сигнала, выделение RR-интервалов и методы анализа RR-сигнала. Описаны из алгоритмы классификации , показавший хорошее качество при классификации периодов времени на сон/бодроствование. Также проведедены эксперименты по предсказанию заболеваний.