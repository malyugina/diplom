\chapter{Основная часть}
\section{Имеющиеся данные}
В рамках программы SAHR были записаны ЭКГ (использовалось только первое отведение) 1800 москвичей преклонного возраста (55-91 год). Из них 46\% мужчин. Запись каждого ЭКГ производилась в течение суток. В это же время за человеком велось дополнительное наблюдение и было известно, в какой промежуток времени он спал, а в какой - бодрствовал. Также про каждого человека известна некоторая информация: состояние здоровья, вес, курит ли он и многое другое.
\section{Постановка задачи}

Цель работы - анализируя данные ЭКГ предсказать в какое время человек спал/бодроствовал, исследовать возможность предсказания болезней и самочувствия человека.Также в рамках обучени выделения признаков и их анализа были проведены исследования по идентификации человека.

\section{Актуальность задачи}
Электрокардиография - самый распространенный клинический инструмент, который измеряет электрическую деятельность сердца с поверхности тела. Сигнал ЭКГ, или проще вариабельности сердечного ритма содержит много интересной информации о человеке. Процесс получения информации об ЭКГ у людей стал простым !!!!!!!!(более технически простым) с изобретения новых технологий и приборов, таких как фитнесс-браслеты, специальные чехлы для телефонов, мониторы-холтеры и т. д. Также создание базы данных, собранных в дополнение к существующим (SAHR, AHA DB, ESC DB и т. д.). в настоящее время.

Процесс сна имеет важное значение для медицинской диагностики. Классическая полисомнография для мониторинга сна накладывает существенные ограничения на практике. В последние годы, носимых устройств для мониторинга здоровья становится все более популярными. Браслеты и смарт-часы позволяют пациенту быть подвижным, в то время как необходимые данные собираются в своей естественной среде. Однако, на сегодняшний день нет хороших научных доказательств, представленных публике, о том, что эти устройства могут оценить продолжительность сна с высокой точностью.

\section{Фильтрация ЭКГ сигнала и выделение RR-пиков}
\subsection{Фильтрация ЭКГ}
К экг сигналу применялся ряд фильтров: фильтры высоких и низких частот. 
тут будет картинки 4.
\subsection{Выделение RR-пиков}
Для определения RR-пиков использовалсz следующий алгоритм.

\begin{enumerate}
	\item На протяжении 200мс бралась точка максимального участока сигнала.
	\item Справа от него в течение 5 мс бралась минимальная точка и проверялось, что это локальний минимум.
	\item Разница высот между минимум и максимумом должна бть больше порогового значения.
	\item Проверялось, что слева от максимумfа в пределах 20 мс также можно найти точку, меньше чем максимум для заданного значения
\end{enumerate}

!!!иллюстрация алгоритма

\section{Работа с RR-сигналом}
\subsection{Отбор RR-пиков}
Для классификации промежутков времени на сон/бодрствование необходим весь временной ряд. Но для идентификации человека, для предсказания каких-либо характеристик достаточно только его части. Если брать для предсказания характеристик небольшие участки сигнала размер обучающей выборки увеличится в несколько раз, увеличится ее обобщающая способность.

То, что мы берем только участки сигнала позволяет нам отобрать участки с минимумом шума. На сигнал могут влиять случайные движения человека, непроизвольные сокращения мыщц. Может быть не очень хорошо закреплен электрод или же он может сдвинуться или отойти на время. в первом случае мы получаем лишние пики, во втором, наоборот, долгий промежуток без них. 

Было решено отбирать участки сигнала где все выделенные R-пики удовлетворяют следующим условиям:

\begin{itemize}
	\item RR интервалы больше 200 милисекунд
	\item RR интервалы меньше 2 секунд
	\item разница высот соседних пиков менее 20\%
	\item разница продолжительностей соседних RR интервалов менее 20\%
\end{itemize}
\subsection{Выделение признаков}
Признаки!много)
Большая часть подсчитываемых признаков по выделенным R пикам описана в обзоре литературы - они являются стандартными при работе в ЭКГ сигналом. Однако вдобавок к ним использовались еще добавочные признаки.
\begin{itemize}
	\item амплитуды пиком и минимумов в QRST-комплексе
	\item разница высот между пиками P, R, T
	\item отношения ввысот P, R, T
	\item средние значения на промежутках PQ, QR, QT
	\item длительнисть различных фаз в QRST-комплексе
	\item площадь под графиком на промеежутках PQ, QS, ST, begin-Q, S-final
\end{itemize}
В медицине утверждается [!! ссыдки], что врачи ориентируются на данные параметры при постановки диагноза.
\section{Проведенные экперименты}
В рамках данного исследовая проводились попытки решения следующих задач.
\subsection{Идентификация человека}
Люди делились на 3 выборки: тестовую, валидационную и обучающую. Экг каждого человека нарезались на отрезки. Состовлялись пары отрезков, для каждой пары устанавливалось из одного сигнала они взяты (1) или из разных (0). Далее пары отрезков подавались на вход сети :
 архитектуру? надо ли вообще. я об этом так себе знаю - это твое исследование было
Качество предсказания на тестовом датасете - 75\%. Количество элементов каждого класса во всех выборках было равным.
\subsection{Предсказание сна}
Предсказание сна - актуальная задача по следующим причинам.
В ходе данного исследования рассмативались следующие методы
\subsubsection{Линейный классификатор}
Если смотреть на RR-сигнал во время сна и бодроствования (две картинки сюда) можно заметить, что во время сна частота пульса уменьшается (достаточно очевидное предположение известное всем). Также на большей части ЭКГ уменьшается дисперсия сигнала во время сна.
Первым предложенным алгоритмом был следующий.
\begin{enumerate}
	\item для сигнала подсчитывались среднее значение и дисперсия
	\item маркируются все точки сигнала на три категории
	\begin{itemize}
		\item точки на половину дисперсии больше среднего значения
		\item точки на половину дисперсии меньше среднего значения
		\item все остальные точки (формулой бы?)
	\end{itemize}
	\item Далее мы ищем наиболее длинную последовательность точек из первой группы, не прерываемую более чем 5 точками из второй в лоюбом ее месте
\end{enumerate}
Количество точек было подобрано на обучающей выборке. (исправить - посмотреть, сколько точно)
Качество предсказания - 75\%.
Картинок сюда.
\subsubsection{Деревья}
Далее было принято решение увеличить количество признаков и кроме среднего и дисперсии посчитать стандартные приз
Далее было решено проверить следующее предположение: при работе с RR сигналом люди считают множество признаков - может быть не просто так. Давайте посчитаем их и на них что-нибудь обучим.
\subsection{Предсказание болезней}